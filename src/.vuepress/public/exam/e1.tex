\documentclass{article}
\usepackage{zhnumber}
\usepackage{geometry}
 \geometry{
 a4paper,
 total={170mm,257mm},
 left=20mm,
 top=20mm,
 }
\usepackage{xeCJK}
\usepackage{enumitem}

% 定义'答问留白'简化命令
\newcommand{\answerspace}[1][5]{\vspace{#1\baselineskip}}
\newcommand{\questionitem}[2][5]{\item#2\answerspace[#1]}

\renewcommand\thesection{\zhnum{section}}

\begin{document}

\section{vim}

\begin{enumerate}[leftmargin=*]

  \questionitem[2] 写出vim中复制行、粘贴行命令。
  \questionitem[2] 写出vim中剪切行命令。
  \questionitem[2] 写出vim中块选择命令。
  \questionitem[2] 写出vim中字符串替换命令。
  \questionitem[2] 写出vim不保存的命令。
  \questionitem[2] 写出vim执行系统命令\verb|echo ok|的命令。
  \questionitem[2] 写出vim中搜索以\verb|aaa|为行头,以数字为行尾的正则表达式。
  \questionitem[2] 写出vim左右和上下分屏的命令。
  \questionitem[2] 写出vim多文件编辑时,切换到下一个文件的命令。
  \questionitem[2] 写出vim多文件编辑时,删除某个打开文件的操作和命令。
  \questionitem[2] 写出vim显示特殊字符的命令。
  \questionitem[2] 写出vim显示行号的命令。

\end{enumerate}

\section{markdown}

\begin{enumerate}[leftmargin=*]

  \questionitem[2] 写出链接(link)和图片(image)语法。
  \questionitem[3] 写出代码块语法。
  \questionitem[5] 写出表格语法。

\end{enumerate}

\end{document}